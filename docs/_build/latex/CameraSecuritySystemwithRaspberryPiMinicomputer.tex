% Generated by Sphinx.
\def\sphinxdocclass{report}
\newif\ifsphinxKeepOldNames \sphinxKeepOldNamestrue
\documentclass[letterpaper,10pt,english,openany]{sphinxmanual}
\usepackage{iftex}

\ifPDFTeX
  \usepackage[utf8]{inputenc}
\fi
\ifdefined\DeclareUnicodeCharacter
  \DeclareUnicodeCharacter{00A0}{\nobreakspace}
\fi
\usepackage{cmap}
\usepackage[T1]{fontenc}
\usepackage{amsmath,amssymb,amstext}
\usepackage{babel}
\usepackage{times}
\usepackage[Bjarne]{fncychap}
\usepackage{longtable}
\usepackage{sphinx}
\usepackage{multirow}
\usepackage{eqparbox}


\addto\captionsenglish{\renewcommand{\figurename}{Fig.\@ }}
\addto\captionsenglish{\renewcommand{\tablename}{Table }}
\SetupFloatingEnvironment{literal-block}{name=Listing }

\addto\extrasenglish{\def\pageautorefname{page}}




\title{Camera Security System with Raspberry Pi Minicomputer Documentation}
\date{May 21, 2018}
\release{0.1}
\author{Marek Vitula}
\newcommand{\sphinxlogo}{}
\renewcommand{\releasename}{Release}
\makeindex

\makeatletter
\def\PYG@reset{\let\PYG@it=\relax \let\PYG@bf=\relax%
    \let\PYG@ul=\relax \let\PYG@tc=\relax%
    \let\PYG@bc=\relax \let\PYG@ff=\relax}
\def\PYG@tok#1{\csname PYG@tok@#1\endcsname}
\def\PYG@toks#1+{\ifx\relax#1\empty\else%
    \PYG@tok{#1}\expandafter\PYG@toks\fi}
\def\PYG@do#1{\PYG@bc{\PYG@tc{\PYG@ul{%
    \PYG@it{\PYG@bf{\PYG@ff{#1}}}}}}}
\def\PYG#1#2{\PYG@reset\PYG@toks#1+\relax+\PYG@do{#2}}

\expandafter\def\csname PYG@tok@nd\endcsname{\let\PYG@bf=\textbf\def\PYG@tc##1{\textcolor[rgb]{0.33,0.33,0.33}{##1}}}
\expandafter\def\csname PYG@tok@sb\endcsname{\def\PYG@tc##1{\textcolor[rgb]{0.25,0.44,0.63}{##1}}}
\expandafter\def\csname PYG@tok@ne\endcsname{\def\PYG@tc##1{\textcolor[rgb]{0.00,0.44,0.13}{##1}}}
\expandafter\def\csname PYG@tok@mo\endcsname{\def\PYG@tc##1{\textcolor[rgb]{0.13,0.50,0.31}{##1}}}
\expandafter\def\csname PYG@tok@mf\endcsname{\def\PYG@tc##1{\textcolor[rgb]{0.13,0.50,0.31}{##1}}}
\expandafter\def\csname PYG@tok@se\endcsname{\let\PYG@bf=\textbf\def\PYG@tc##1{\textcolor[rgb]{0.25,0.44,0.63}{##1}}}
\expandafter\def\csname PYG@tok@ow\endcsname{\let\PYG@bf=\textbf\def\PYG@tc##1{\textcolor[rgb]{0.00,0.44,0.13}{##1}}}
\expandafter\def\csname PYG@tok@gp\endcsname{\let\PYG@bf=\textbf\def\PYG@tc##1{\textcolor[rgb]{0.78,0.36,0.04}{##1}}}
\expandafter\def\csname PYG@tok@il\endcsname{\def\PYG@tc##1{\textcolor[rgb]{0.13,0.50,0.31}{##1}}}
\expandafter\def\csname PYG@tok@vc\endcsname{\def\PYG@tc##1{\textcolor[rgb]{0.73,0.38,0.84}{##1}}}
\expandafter\def\csname PYG@tok@mi\endcsname{\def\PYG@tc##1{\textcolor[rgb]{0.13,0.50,0.31}{##1}}}
\expandafter\def\csname PYG@tok@nl\endcsname{\let\PYG@bf=\textbf\def\PYG@tc##1{\textcolor[rgb]{0.00,0.13,0.44}{##1}}}
\expandafter\def\csname PYG@tok@si\endcsname{\let\PYG@it=\textit\def\PYG@tc##1{\textcolor[rgb]{0.44,0.63,0.82}{##1}}}
\expandafter\def\csname PYG@tok@s2\endcsname{\def\PYG@tc##1{\textcolor[rgb]{0.25,0.44,0.63}{##1}}}
\expandafter\def\csname PYG@tok@kn\endcsname{\let\PYG@bf=\textbf\def\PYG@tc##1{\textcolor[rgb]{0.00,0.44,0.13}{##1}}}
\expandafter\def\csname PYG@tok@gt\endcsname{\def\PYG@tc##1{\textcolor[rgb]{0.00,0.27,0.87}{##1}}}
\expandafter\def\csname PYG@tok@mh\endcsname{\def\PYG@tc##1{\textcolor[rgb]{0.13,0.50,0.31}{##1}}}
\expandafter\def\csname PYG@tok@o\endcsname{\def\PYG@tc##1{\textcolor[rgb]{0.40,0.40,0.40}{##1}}}
\expandafter\def\csname PYG@tok@vg\endcsname{\def\PYG@tc##1{\textcolor[rgb]{0.73,0.38,0.84}{##1}}}
\expandafter\def\csname PYG@tok@fm\endcsname{\def\PYG@tc##1{\textcolor[rgb]{0.02,0.16,0.49}{##1}}}
\expandafter\def\csname PYG@tok@kt\endcsname{\def\PYG@tc##1{\textcolor[rgb]{0.56,0.13,0.00}{##1}}}
\expandafter\def\csname PYG@tok@gh\endcsname{\let\PYG@bf=\textbf\def\PYG@tc##1{\textcolor[rgb]{0.00,0.00,0.50}{##1}}}
\expandafter\def\csname PYG@tok@nf\endcsname{\def\PYG@tc##1{\textcolor[rgb]{0.02,0.16,0.49}{##1}}}
\expandafter\def\csname PYG@tok@sr\endcsname{\def\PYG@tc##1{\textcolor[rgb]{0.14,0.33,0.53}{##1}}}
\expandafter\def\csname PYG@tok@ch\endcsname{\let\PYG@it=\textit\def\PYG@tc##1{\textcolor[rgb]{0.25,0.50,0.56}{##1}}}
\expandafter\def\csname PYG@tok@no\endcsname{\def\PYG@tc##1{\textcolor[rgb]{0.38,0.68,0.84}{##1}}}
\expandafter\def\csname PYG@tok@gd\endcsname{\def\PYG@tc##1{\textcolor[rgb]{0.63,0.00,0.00}{##1}}}
\expandafter\def\csname PYG@tok@ni\endcsname{\let\PYG@bf=\textbf\def\PYG@tc##1{\textcolor[rgb]{0.84,0.33,0.22}{##1}}}
\expandafter\def\csname PYG@tok@nc\endcsname{\let\PYG@bf=\textbf\def\PYG@tc##1{\textcolor[rgb]{0.05,0.52,0.71}{##1}}}
\expandafter\def\csname PYG@tok@nv\endcsname{\def\PYG@tc##1{\textcolor[rgb]{0.73,0.38,0.84}{##1}}}
\expandafter\def\csname PYG@tok@k\endcsname{\let\PYG@bf=\textbf\def\PYG@tc##1{\textcolor[rgb]{0.00,0.44,0.13}{##1}}}
\expandafter\def\csname PYG@tok@sx\endcsname{\def\PYG@tc##1{\textcolor[rgb]{0.78,0.36,0.04}{##1}}}
\expandafter\def\csname PYG@tok@dl\endcsname{\def\PYG@tc##1{\textcolor[rgb]{0.25,0.44,0.63}{##1}}}
\expandafter\def\csname PYG@tok@nb\endcsname{\def\PYG@tc##1{\textcolor[rgb]{0.00,0.44,0.13}{##1}}}
\expandafter\def\csname PYG@tok@gs\endcsname{\let\PYG@bf=\textbf}
\expandafter\def\csname PYG@tok@err\endcsname{\def\PYG@bc##1{\setlength{\fboxsep}{0pt}\fcolorbox[rgb]{1.00,0.00,0.00}{1,1,1}{\strut ##1}}}
\expandafter\def\csname PYG@tok@cs\endcsname{\def\PYG@tc##1{\textcolor[rgb]{0.25,0.50,0.56}{##1}}\def\PYG@bc##1{\setlength{\fboxsep}{0pt}\colorbox[rgb]{1.00,0.94,0.94}{\strut ##1}}}
\expandafter\def\csname PYG@tok@sc\endcsname{\def\PYG@tc##1{\textcolor[rgb]{0.25,0.44,0.63}{##1}}}
\expandafter\def\csname PYG@tok@cpf\endcsname{\let\PYG@it=\textit\def\PYG@tc##1{\textcolor[rgb]{0.25,0.50,0.56}{##1}}}
\expandafter\def\csname PYG@tok@c1\endcsname{\let\PYG@it=\textit\def\PYG@tc##1{\textcolor[rgb]{0.25,0.50,0.56}{##1}}}
\expandafter\def\csname PYG@tok@kd\endcsname{\let\PYG@bf=\textbf\def\PYG@tc##1{\textcolor[rgb]{0.00,0.44,0.13}{##1}}}
\expandafter\def\csname PYG@tok@cm\endcsname{\let\PYG@it=\textit\def\PYG@tc##1{\textcolor[rgb]{0.25,0.50,0.56}{##1}}}
\expandafter\def\csname PYG@tok@mb\endcsname{\def\PYG@tc##1{\textcolor[rgb]{0.13,0.50,0.31}{##1}}}
\expandafter\def\csname PYG@tok@gr\endcsname{\def\PYG@tc##1{\textcolor[rgb]{1.00,0.00,0.00}{##1}}}
\expandafter\def\csname PYG@tok@c\endcsname{\let\PYG@it=\textit\def\PYG@tc##1{\textcolor[rgb]{0.25,0.50,0.56}{##1}}}
\expandafter\def\csname PYG@tok@nn\endcsname{\let\PYG@bf=\textbf\def\PYG@tc##1{\textcolor[rgb]{0.05,0.52,0.71}{##1}}}
\expandafter\def\csname PYG@tok@gi\endcsname{\def\PYG@tc##1{\textcolor[rgb]{0.00,0.63,0.00}{##1}}}
\expandafter\def\csname PYG@tok@na\endcsname{\def\PYG@tc##1{\textcolor[rgb]{0.25,0.44,0.63}{##1}}}
\expandafter\def\csname PYG@tok@sa\endcsname{\def\PYG@tc##1{\textcolor[rgb]{0.25,0.44,0.63}{##1}}}
\expandafter\def\csname PYG@tok@kc\endcsname{\let\PYG@bf=\textbf\def\PYG@tc##1{\textcolor[rgb]{0.00,0.44,0.13}{##1}}}
\expandafter\def\csname PYG@tok@kp\endcsname{\def\PYG@tc##1{\textcolor[rgb]{0.00,0.44,0.13}{##1}}}
\expandafter\def\csname PYG@tok@s\endcsname{\def\PYG@tc##1{\textcolor[rgb]{0.25,0.44,0.63}{##1}}}
\expandafter\def\csname PYG@tok@sh\endcsname{\def\PYG@tc##1{\textcolor[rgb]{0.25,0.44,0.63}{##1}}}
\expandafter\def\csname PYG@tok@sd\endcsname{\let\PYG@it=\textit\def\PYG@tc##1{\textcolor[rgb]{0.25,0.44,0.63}{##1}}}
\expandafter\def\csname PYG@tok@ge\endcsname{\let\PYG@it=\textit}
\expandafter\def\csname PYG@tok@cp\endcsname{\def\PYG@tc##1{\textcolor[rgb]{0.00,0.44,0.13}{##1}}}
\expandafter\def\csname PYG@tok@vm\endcsname{\def\PYG@tc##1{\textcolor[rgb]{0.73,0.38,0.84}{##1}}}
\expandafter\def\csname PYG@tok@gu\endcsname{\let\PYG@bf=\textbf\def\PYG@tc##1{\textcolor[rgb]{0.50,0.00,0.50}{##1}}}
\expandafter\def\csname PYG@tok@vi\endcsname{\def\PYG@tc##1{\textcolor[rgb]{0.73,0.38,0.84}{##1}}}
\expandafter\def\csname PYG@tok@m\endcsname{\def\PYG@tc##1{\textcolor[rgb]{0.13,0.50,0.31}{##1}}}
\expandafter\def\csname PYG@tok@w\endcsname{\def\PYG@tc##1{\textcolor[rgb]{0.73,0.73,0.73}{##1}}}
\expandafter\def\csname PYG@tok@s1\endcsname{\def\PYG@tc##1{\textcolor[rgb]{0.25,0.44,0.63}{##1}}}
\expandafter\def\csname PYG@tok@ss\endcsname{\def\PYG@tc##1{\textcolor[rgb]{0.32,0.47,0.09}{##1}}}
\expandafter\def\csname PYG@tok@bp\endcsname{\def\PYG@tc##1{\textcolor[rgb]{0.00,0.44,0.13}{##1}}}
\expandafter\def\csname PYG@tok@go\endcsname{\def\PYG@tc##1{\textcolor[rgb]{0.20,0.20,0.20}{##1}}}
\expandafter\def\csname PYG@tok@kr\endcsname{\let\PYG@bf=\textbf\def\PYG@tc##1{\textcolor[rgb]{0.00,0.44,0.13}{##1}}}
\expandafter\def\csname PYG@tok@nt\endcsname{\let\PYG@bf=\textbf\def\PYG@tc##1{\textcolor[rgb]{0.02,0.16,0.45}{##1}}}

\def\PYGZbs{\char`\\}
\def\PYGZus{\char`\_}
\def\PYGZob{\char`\{}
\def\PYGZcb{\char`\}}
\def\PYGZca{\char`\^}
\def\PYGZam{\char`\&}
\def\PYGZlt{\char`\<}
\def\PYGZgt{\char`\>}
\def\PYGZsh{\char`\#}
\def\PYGZpc{\char`\%}
\def\PYGZdl{\char`\$}
\def\PYGZhy{\char`\-}
\def\PYGZsq{\char`\'}
\def\PYGZdq{\char`\"}
\def\PYGZti{\char`\~}
% for compatibility with earlier versions
\def\PYGZat{@}
\def\PYGZlb{[}
\def\PYGZrb{]}
\makeatother

\renewcommand\PYGZsq{\textquotesingle}

\begin{document}

\maketitle
\tableofcontents
\phantomsection\label{rpicameramon::doc}


This package provides camera security system functionality for the Raspberry Pi minicomputer.
It uses Raspberry Pi NoIR camera for the motion detection and also PIR sensor if available.

Capturing is handled by the picamera package and image processing is handled by the Pillow package.

Captured images are sent to the Google Drive storage.


\chapter{Submodules}
\label{rpicameramon:submodules}\label{rpicameramon:rpicameramon-package}
Package itself consist of four submodules - config, filemanipulation, motion and telemetry.
Each of them has consist of several classes.


\chapter{config module}
\label{rpicameramon:config-module}
Config modules is used for storing of the application configuration internally.
It consist of two classes - BaseConfig and UserConfig. UserConfig inherits from the BaseConfig class.

UserConfig class adds variables which can be loaded from the external JSON file.
\phantomsection\label{rpicameramon:module-rpicameramon.config}\index{rpicameramon.config (module)}\index{BaseConfig (class in rpicameramon.config)}

\begin{fulllineitems}
\phantomsection\label{rpicameramon:rpicameramon.config.BaseConfig}\pysigline{\sphinxstrong{class }\sphinxcode{rpicameramon.config.}\sphinxbfcode{BaseConfig}}
Bases: \sphinxcode{object}

Basic configuration which cannot be changed online
\index{imageDir (rpicameramon.config.BaseConfig attribute)}

\begin{fulllineitems}
\phantomsection\label{rpicameramon:rpicameramon.config.BaseConfig.imageDir}\pysigline{\sphinxbfcode{imageDir}}
\emph{str} -- default image directory

\end{fulllineitems}

\index{imagePath (rpicameramon.config.BaseConfig attribute)}

\begin{fulllineitems}
\phantomsection\label{rpicameramon:rpicameramon.config.BaseConfig.imagePath}\pysigline{\sphinxbfcode{imagePath}}
\emph{str} -- path to the directory + imageDir

\end{fulllineitems}

\index{imageWidth (rpicameramon.config.BaseConfig attribute)}

\begin{fulllineitems}
\phantomsection\label{rpicameramon:rpicameramon.config.BaseConfig.imageWidth}\pysigline{\sphinxbfcode{imageWidth}}
\emph{int} -- width of taken pictures in pixels

\end{fulllineitems}

\index{imageHeight (rpicameramon.config.BaseConfig attribute)}

\begin{fulllineitems}
\phantomsection\label{rpicameramon:rpicameramon.config.BaseConfig.imageHeight}\pysigline{\sphinxbfcode{imageHeight}}
\emph{int} -- height of taken pictures in pixels

\end{fulllineitems}

\index{imageVFlip (rpicameramon.config.BaseConfig attribute)}

\begin{fulllineitems}
\phantomsection\label{rpicameramon:rpicameramon.config.BaseConfig.imageVFlip}\pysigline{\sphinxbfcode{imageVFlip}}
\emph{bool} -- flips image vertically if true

\end{fulllineitems}

\index{imageHFlip (rpicameramon.config.BaseConfig attribute)}

\begin{fulllineitems}
\phantomsection\label{rpicameramon:rpicameramon.config.BaseConfig.imageHFlip}\pysigline{\sphinxbfcode{imageHFlip}}
\emph{bool} -- flips image Horizontally if true

\end{fulllineitems}

\index{imagePreview (rpicameramon.config.BaseConfig attribute)}

\begin{fulllineitems}
\phantomsection\label{rpicameramon:rpicameramon.config.BaseConfig.imagePreview}\pysigline{\sphinxbfcode{imagePreview}}
\emph{bool} -- opens window with picture if true

\end{fulllineitems}

\index{imageQuality (rpicameramon.config.BaseConfig attribute)}

\begin{fulllineitems}
\phantomsection\label{rpicameramon:rpicameramon.config.BaseConfig.imageQuality}\pysigline{\sphinxbfcode{imageQuality}}
\emph{int} -- JPEG quality 0...100

\end{fulllineitems}

\index{vecMagnitude (rpicameramon.config.BaseConfig attribute)}

\begin{fulllineitems}
\phantomsection\label{rpicameramon:rpicameramon.config.BaseConfig.vecMagnitude}\pysigline{\sphinxbfcode{vecMagnitude}}
\emph{int} -- minimum magnitude of a motion vector

\end{fulllineitems}

\index{vecCount (rpicameramon.config.BaseConfig attribute)}

\begin{fulllineitems}
\phantomsection\label{rpicameramon:rpicameramon.config.BaseConfig.vecCount}\pysigline{\sphinxbfcode{vecCount}}
\emph{int} -- minimal number of motion vectors

\end{fulllineitems}

\index{cameraFPS (rpicameramon.config.BaseConfig attribute)}

\begin{fulllineitems}
\phantomsection\label{rpicameramon:rpicameramon.config.BaseConfig.cameraFPS}\pysigline{\sphinxbfcode{cameraFPS}}
\emph{int} -- framerate of the camera

\end{fulllineitems}

\index{confCheckTime (rpicameramon.config.BaseConfig attribute)}

\begin{fulllineitems}
\phantomsection\label{rpicameramon:rpicameramon.config.BaseConfig.confCheckTime}\pysigline{\sphinxbfcode{confCheckTime}}
\emph{int} -- how long does it take between conf file checks in sec

\end{fulllineitems}

\index{PIRpin (rpicameramon.config.BaseConfig attribute)}

\begin{fulllineitems}
\phantomsection\label{rpicameramon:rpicameramon.config.BaseConfig.PIRpin}\pysigline{\sphinxbfcode{PIRpin}}
\emph{int} -- GPIO pin (BCM) where is PIR sensor connected

\end{fulllineitems}

\index{LEDpin (rpicameramon.config.BaseConfig attribute)}

\begin{fulllineitems}
\phantomsection\label{rpicameramon:rpicameramon.config.BaseConfig.LEDpin}\pysigline{\sphinxbfcode{LEDpin}}
\emph{int} -- GPIO pin (BCM) where is LED connected

\end{fulllineitems}

\index{pictureFolderID (rpicameramon.config.BaseConfig attribute)}

\begin{fulllineitems}
\phantomsection\label{rpicameramon:rpicameramon.config.BaseConfig.pictureFolderID}\pysigline{\sphinxbfcode{pictureFolderID}}
\emph{str} -- Google Drive ID of folder with pictures

\end{fulllineitems}

\index{confFileID (rpicameramon.config.BaseConfig attribute)}

\begin{fulllineitems}
\phantomsection\label{rpicameramon:rpicameramon.config.BaseConfig.confFileID}\pysigline{\sphinxbfcode{confFileID}}
\emph{str} -- Google Drive ID of JSON configuration files

\end{fulllineitems}

\index{dashboardFileID (rpicameramon.config.BaseConfig attribute)}

\begin{fulllineitems}
\phantomsection\label{rpicameramon:rpicameramon.config.BaseConfig.dashboardFileID}\pysigline{\sphinxbfcode{dashboardFileID}}
\emph{str} -- Google Drive ID of dashboard panel

\end{fulllineitems}

\index{logRange (rpicameramon.config.BaseConfig attribute)}

\begin{fulllineitems}
\phantomsection\label{rpicameramon:rpicameramon.config.BaseConfig.logRange}\pysigline{\sphinxbfcode{logRange}}
\emph{str} -- Range in sheets for telemetry messages

\end{fulllineitems}

\index{msgRange (rpicameramon.config.BaseConfig attribute)}

\begin{fulllineitems}
\phantomsection\label{rpicameramon:rpicameramon.config.BaseConfig.msgRange}\pysigline{\sphinxbfcode{msgRange}}
\emph{str} -- Range in sheets for log messages

\end{fulllineitems}

\index{OAuthJSON (rpicameramon.config.BaseConfig attribute)}

\begin{fulllineitems}
\phantomsection\label{rpicameramon:rpicameramon.config.BaseConfig.OAuthJSON}\pysigline{\sphinxbfcode{OAuthJSON}}
\emph{str} -- Name of the JSON file for Google OAuth authentication

\end{fulllineitems}

\index{fileUploadSleep (rpicameramon.config.BaseConfig attribute)}

\begin{fulllineitems}
\phantomsection\label{rpicameramon:rpicameramon.config.BaseConfig.fileUploadSleep}\pysigline{\sphinxbfcode{fileUploadSleep}}
\emph{int} -- Sleep time in seconds for file uploader thread

\end{fulllineitems}

\index{batchUploadWindow (rpicameramon.config.BaseConfig attribute)}

\begin{fulllineitems}
\phantomsection\label{rpicameramon:rpicameramon.config.BaseConfig.batchUploadWindow}\pysigline{\sphinxbfcode{batchUploadWindow}}
\emph{int} -- Allowed hour for file uploads in batch mode

\end{fulllineitems}

\index{GSMport (rpicameramon.config.BaseConfig attribute)}

\begin{fulllineitems}
\phantomsection\label{rpicameramon:rpicameramon.config.BaseConfig.GSMport}\pysigline{\sphinxbfcode{GSMport}}
\emph{str} -- Device address of a GSM Module

\end{fulllineitems}

\index{GSMbaud (rpicameramon.config.BaseConfig attribute)}

\begin{fulllineitems}
\phantomsection\label{rpicameramon:rpicameramon.config.BaseConfig.GSMbaud}\pysigline{\sphinxbfcode{GSMbaud}}
\emph{int} -- Baud rate for communication with the GSM module

\end{fulllineitems}

\index{GSMPIN (rpicameramon.config.BaseConfig attribute)}

\begin{fulllineitems}
\phantomsection\label{rpicameramon:rpicameramon.config.BaseConfig.GSMPIN}\pysigline{\sphinxbfcode{GSMPIN}}
\emph{int} -- PIN for unlocking the SIM card

\end{fulllineitems}


\end{fulllineitems}

\index{UserConfig (class in rpicameramon.config)}

\begin{fulllineitems}
\phantomsection\label{rpicameramon:rpicameramon.config.UserConfig}\pysigline{\sphinxstrong{class }\sphinxcode{rpicameramon.config.}\sphinxbfcode{UserConfig}}
Bases: {\hyperref[rpicameramon:rpicameramon.config.BaseConfig]{\sphinxcrossref{\sphinxcode{rpicameramon.config.BaseConfig}}}}

Extends BaseConfig class with the user defined configuration.
This configuration can be updated from the external JSON.
\index{mode (rpicameramon.config.UserConfig attribute)}

\begin{fulllineitems}
\phantomsection\label{rpicameramon:rpicameramon.config.UserConfig.mode}\pysigline{\sphinxbfcode{mode}}
\emph{str} -- default runtime mode (realtime, interval, batch, ondemand)

\end{fulllineitems}

\index{echo (rpicameramon.config.UserConfig attribute)}

\begin{fulllineitems}
\phantomsection\label{rpicameramon:rpicameramon.config.UserConfig.echo}\pysigline{\sphinxbfcode{echo}}
\emph{boolean} -- echo mode

\end{fulllineitems}

\index{interval (rpicameramon.config.UserConfig attribute)}

\begin{fulllineitems}
\phantomsection\label{rpicameramon:rpicameramon.config.UserConfig.interval}\pysigline{\sphinxbfcode{interval}}
\emph{int} -- default interval time in seconds for interval mode

\end{fulllineitems}

\index{storage (rpicameramon.config.UserConfig attribute)}

\begin{fulllineitems}
\phantomsection\label{rpicameramon:rpicameramon.config.UserConfig.storage}\pysigline{\sphinxbfcode{storage}}
\emph{str} -- gdrive or dropbox. Dropbox is only experimental

\end{fulllineitems}

\index{usePIR (rpicameramon.config.UserConfig attribute)}

\begin{fulllineitems}
\phantomsection\label{rpicameramon:rpicameramon.config.UserConfig.usePIR}\pysigline{\sphinxbfcode{usePIR}}
\emph{boolean} -- use PIR sensor

\end{fulllineitems}

\index{SMSNotification (rpicameramon.config.UserConfig attribute)}

\begin{fulllineitems}
\phantomsection\label{rpicameramon:rpicameramon.config.UserConfig.SMSNotification}\pysigline{\sphinxbfcode{SMSNotification}}
\emph{boolean} -- SMS SMSNotification

\end{fulllineitems}

\index{SMSControl (rpicameramon.config.UserConfig attribute)}

\begin{fulllineitems}
\phantomsection\label{rpicameramon:rpicameramon.config.UserConfig.SMSControl}\pysigline{\sphinxbfcode{SMSControl}}
\emph{boolean} -- Allow control via SMS messages

\end{fulllineitems}

\index{authorizedNumber (rpicameramon.config.UserConfig attribute)}

\begin{fulllineitems}
\phantomsection\label{rpicameramon:rpicameramon.config.UserConfig.authorizedNumber}\pysigline{\sphinxbfcode{authorizedNumber}}
\emph{int} -- Authorized number for control and notifications

\end{fulllineitems}

\index{load\_config() (rpicameramon.config.UserConfig class method)}

\begin{fulllineitems}
\phantomsection\label{rpicameramon:rpicameramon.config.UserConfig.load_config}\pysiglinewithargsret{\sphinxstrong{classmethod }\sphinxbfcode{load\_config}}{\emph{conf}}{}
This method loads configuration to UserConfig class from the JSON file.
\begin{quote}\begin{description}
\item[{Raises}] \leavevmode
\sphinxcode{KeyError} -- If JSON cannot be parsed

\end{description}\end{quote}

\end{fulllineitems}


\end{fulllineitems}



\chapter{filemanipulation module}
\label{rpicameramon:filemanipulation-module}
This module consist of classes for the manipulation with files in Google Drive.
\phantomsection\label{rpicameramon:module-rpicameramon.filemanipulation}\index{rpicameramon.filemanipulation (module)}\index{ConfFileDownloader (class in rpicameramon.filemanipulation)}

\begin{fulllineitems}
\phantomsection\label{rpicameramon:rpicameramon.filemanipulation.ConfFileDownloader}\pysiglinewithargsret{\sphinxstrong{class }\sphinxcode{rpicameramon.filemanipulation.}\sphinxbfcode{ConfFileDownloader}}{\emph{group=None}, \emph{target=None}, \emph{name=None}, \emph{filename=None}, \emph{*}, \emph{daemon=None}}{}
Bases: \sphinxcode{threading.Thread}

ConfFileDownloader is a subclass of a Thread class
First it loads the initial JSON configuration from Google drive
and then it check if the configuration was changed. If yes, it loads it.
\index{filename (rpicameramon.filemanipulation.ConfFileDownloader attribute)}

\begin{fulllineitems}
\phantomsection\label{rpicameramon:rpicameramon.filemanipulation.ConfFileDownloader.filename}\pysigline{\sphinxbfcode{filename}}
\emph{string} -- file ID of JSON configuration file on Google drive

\end{fulllineitems}

\index{run() (rpicameramon.filemanipulation.ConfFileDownloader method)}

\begin{fulllineitems}
\phantomsection\label{rpicameramon:rpicameramon.filemanipulation.ConfFileDownloader.run}\pysiglinewithargsret{\sphinxbfcode{run}}{}{}
This method runs when ConfFileDownloader thread is started.
First it initialize connection with Google Drive and starts loop
for checking if JSON configuration file on Google Drive was modified.
If yes, It downloads it and calls the method for loading it into UserConfig class.

File is checked in intervals specified in confCheckTime variable.

\end{fulllineitems}

\index{auth() (rpicameramon.filemanipulation.ConfFileDownloader method)}

\begin{fulllineitems}
\phantomsection\label{rpicameramon:rpicameramon.filemanipulation.ConfFileDownloader.auth}\pysiglinewithargsret{\sphinxbfcode{auth}}{}{}
Function for Google Drive authentication. It loads credentials from the
file ``mycreds.txt''.
\begin{quote}\begin{description}
\item[{Returns}] \leavevmode
Drive object

\item[{Return type}] \leavevmode
drive

\end{description}\end{quote}

\end{fulllineitems}

\index{download\_file() (rpicameramon.filemanipulation.ConfFileDownloader method)}

\begin{fulllineitems}
\phantomsection\label{rpicameramon:rpicameramon.filemanipulation.ConfFileDownloader.download_file}\pysiglinewithargsret{\sphinxbfcode{download\_file}}{\emph{drive}}{}
This function downloads specified file from the Google Drive.
\begin{quote}\begin{description}
\item[{Parameters}] \leavevmode
\textbf{\texttt{drive}} (\emph{\texttt{obj}}) -- Google Drive object

\item[{Returns}] \leavevmode
Content of the file as string

\item[{Return type}] \leavevmode
str

\end{description}\end{quote}

\end{fulllineitems}

\index{get\_timestamp() (rpicameramon.filemanipulation.ConfFileDownloader method)}

\begin{fulllineitems}
\phantomsection\label{rpicameramon:rpicameramon.filemanipulation.ConfFileDownloader.get_timestamp}\pysiglinewithargsret{\sphinxbfcode{get\_timestamp}}{\emph{drive}}{}
This function fetches date of the last modification of specified file.
\begin{quote}\begin{description}
\item[{Returns}] \leavevmode
Timestamp of last modification of the file

\item[{Return type}] \leavevmode
timestamp

\end{description}\end{quote}

\end{fulllineitems}

\index{parse\_json() (rpicameramon.filemanipulation.ConfFileDownloader method)}

\begin{fulllineitems}
\phantomsection\label{rpicameramon:rpicameramon.filemanipulation.ConfFileDownloader.parse_json}\pysiglinewithargsret{\sphinxbfcode{parse\_json}}{\emph{jsonfile}}{}
\end{fulllineitems}

\index{load\_config() (rpicameramon.filemanipulation.ConfFileDownloader method)}

\begin{fulllineitems}
\phantomsection\label{rpicameramon:rpicameramon.filemanipulation.ConfFileDownloader.load_config}\pysiglinewithargsret{\sphinxbfcode{load\_config}}{\emph{configdata}}{}
\end{fulllineitems}


\end{fulllineitems}

\index{FileUploader (class in rpicameramon.filemanipulation)}

\begin{fulllineitems}
\phantomsection\label{rpicameramon:rpicameramon.filemanipulation.FileUploader}\pysiglinewithargsret{\sphinxstrong{class }\sphinxcode{rpicameramon.filemanipulation.}\sphinxbfcode{FileUploader}}{\emph{group=None}, \emph{target=None}, \emph{name=None}, \emph{storage=None}, \emph{*}, \emph{daemon=None}, \emph{q=None}}{}
Bases: \sphinxcode{threading.Thread}

This class provides a File uploader thread for uploading
captured photos. It supports google drive and dropbox (experimental)
\index{storage (rpicameramon.filemanipulation.FileUploader attribute)}

\begin{fulllineitems}
\phantomsection\label{rpicameramon:rpicameramon.filemanipulation.FileUploader.storage}\pysigline{\sphinxbfcode{storage}}
\emph{str} -- storage type (gdrive or dropbox)

\end{fulllineitems}

\index{q (rpicameramon.filemanipulation.FileUploader attribute)}

\begin{fulllineitems}
\phantomsection\label{rpicameramon:rpicameramon.filemanipulation.FileUploader.q}\pysigline{\sphinxbfcode{q}}
\emph{Queue obj} -- queue with filenames of pictures to be uploaded

\end{fulllineitems}

\index{dropbox\_is\_init (rpicameramon.filemanipulation.FileUploader attribute)}

\begin{fulllineitems}
\phantomsection\label{rpicameramon:rpicameramon.filemanipulation.FileUploader.dropbox_is_init}\pysigline{\sphinxbfcode{dropbox\_is\_init}}
\emph{boolean} -- True if Dropbox is initialized

\end{fulllineitems}

\index{gdrive\_is\_init (rpicameramon.filemanipulation.FileUploader attribute)}

\begin{fulllineitems}
\phantomsection\label{rpicameramon:rpicameramon.filemanipulation.FileUploader.gdrive_is_init}\pysigline{\sphinxbfcode{gdrive\_is\_init}}
\emph{boolean} -- True if Google Drive is initialized

\end{fulllineitems}

\index{run() (rpicameramon.filemanipulation.FileUploader method)}

\begin{fulllineitems}
\phantomsection\label{rpicameramon:rpicameramon.filemanipulation.FileUploader.run}\pysiglinewithargsret{\sphinxbfcode{run}}{}{}
\end{fulllineitems}

\index{gdrive\_init() (rpicameramon.filemanipulation.FileUploader method)}

\begin{fulllineitems}
\phantomsection\label{rpicameramon:rpicameramon.filemanipulation.FileUploader.gdrive_init}\pysiglinewithargsret{\sphinxbfcode{gdrive\_init}}{}{}
\end{fulllineitems}

\index{gdrive\_upload() (rpicameramon.filemanipulation.FileUploader method)}

\begin{fulllineitems}
\phantomsection\label{rpicameramon:rpicameramon.filemanipulation.FileUploader.gdrive_upload}\pysiglinewithargsret{\sphinxbfcode{gdrive\_upload}}{\emph{filename}, \emph{drive}}{}
\end{fulllineitems}

\index{dropbox\_init() (rpicameramon.filemanipulation.FileUploader method)}

\begin{fulllineitems}
\phantomsection\label{rpicameramon:rpicameramon.filemanipulation.FileUploader.dropbox_init}\pysiglinewithargsret{\sphinxbfcode{dropbox\_init}}{}{}
\end{fulllineitems}

\index{dropbox\_upload() (rpicameramon.filemanipulation.FileUploader method)}

\begin{fulllineitems}
\phantomsection\label{rpicameramon:rpicameramon.filemanipulation.FileUploader.dropbox_upload}\pysiglinewithargsret{\sphinxbfcode{dropbox\_upload}}{\emph{filename}, \emph{dbx}}{}
\end{fulllineitems}


\end{fulllineitems}

\index{stopwatch() (in module rpicameramon.filemanipulation)}

\begin{fulllineitems}
\phantomsection\label{rpicameramon:rpicameramon.filemanipulation.stopwatch}\pysiglinewithargsret{\sphinxcode{rpicameramon.filemanipulation.}\sphinxbfcode{stopwatch}}{\emph{message}}{}
\end{fulllineitems}



\chapter{motion module}
\label{rpicameramon:motion-module}
This module has functionality for the motion detection from the scene and also from the PIR sensor
and also SMS handler.
Optical flow method is used for the motion detection.
\phantomsection\label{rpicameramon:module-rpicameramon.motion}\index{rpicameramon.motion (module)}\index{MotionAnalysis (class in rpicameramon.motion)}

\begin{fulllineitems}
\phantomsection\label{rpicameramon:rpicameramon.motion.MotionAnalysis}\pysiglinewithargsret{\sphinxstrong{class }\sphinxcode{rpicameramon.motion.}\sphinxbfcode{MotionAnalysis}}{\emph{camera}, \emph{handler}}{}
Bases: \sphinxcode{picamera.array.PiMotionAnalysis}

MotionAnalysis class extends PiMotionAnalysis class.

The array passed to analyse() is organized as (rows, columns)
where rows and columns are the number of rows and columns of macro-
blocks (16x16 pixel blocks) in the original frames. There is always
one extra column of macro-blocks present in motion vector data.
\index{analyse() (rpicameramon.motion.MotionAnalysis method)}

\begin{fulllineitems}
\phantomsection\label{rpicameramon:rpicameramon.motion.MotionAnalysis.analyse}\pysiglinewithargsret{\sphinxbfcode{analyse}}{\emph{a}}{}
\end{fulllineitems}


\end{fulllineitems}

\index{PIRMotionAnalysis (class in rpicameramon.motion)}

\begin{fulllineitems}
\phantomsection\label{rpicameramon:rpicameramon.motion.PIRMotionAnalysis}\pysiglinewithargsret{\sphinxstrong{class }\sphinxcode{rpicameramon.motion.}\sphinxbfcode{PIRMotionAnalysis}}{\emph{pin}, \emph{handler}}{}
Bases: \sphinxcode{object}

This class provides a handler for PIR sensor
It detects if the input pin is in the high state
and calls motion\_detected handler
\begin{quote}\begin{description}
\item[{Parameters}] \leavevmode\begin{itemize}
\item {} 
\textbf{\texttt{pin}} (\emph{\texttt{number}}) -- BCM number of GPIO pin where is PIR sensor connected

\item {} 
\textbf{\texttt{handler}} (\emph{\texttt{obj}}) -- CaptureHandler object

\end{itemize}

\end{description}\end{quote}
\index{is\_detected() (rpicameramon.motion.PIRMotionAnalysis method)}

\begin{fulllineitems}
\phantomsection\label{rpicameramon:rpicameramon.motion.PIRMotionAnalysis.is_detected}\pysiglinewithargsret{\sphinxbfcode{is\_detected}}{}{}
\end{fulllineitems}


\end{fulllineitems}

\index{CaptureHandler (class in rpicameramon.motion)}

\begin{fulllineitems}
\phantomsection\label{rpicameramon:rpicameramon.motion.CaptureHandler}\pysiglinewithargsret{\sphinxstrong{class }\sphinxcode{rpicameramon.motion.}\sphinxbfcode{CaptureHandler}}{\emph{camera}, \emph{post\_capture\_callback=None}, \emph{q=None}}{}
Bases: \sphinxcode{object}

It provides a handler for capturing the pictures. With the LED Switch
functionality.
\index{camera (rpicameramon.motion.CaptureHandler attribute)}

\begin{fulllineitems}
\phantomsection\label{rpicameramon:rpicameramon.motion.CaptureHandler.camera}\pysigline{\sphinxbfcode{camera}}
\emph{obj} -- PiCamera object

\end{fulllineitems}

\index{callback (rpicameramon.motion.CaptureHandler attribute)}

\begin{fulllineitems}
\phantomsection\label{rpicameramon:rpicameramon.motion.CaptureHandler.callback}\pysigline{\sphinxbfcode{callback}}
\emph{str} -- callback

\end{fulllineitems}

\index{q (rpicameramon.motion.CaptureHandler attribute)}

\begin{fulllineitems}
\phantomsection\label{rpicameramon:rpicameramon.motion.CaptureHandler.q}\pysigline{\sphinxbfcode{q}}
\emph{obj} -- queue for passing captured photos

\end{fulllineitems}

\index{detected (rpicameramon.motion.CaptureHandler attribute)}

\begin{fulllineitems}
\phantomsection\label{rpicameramon:rpicameramon.motion.CaptureHandler.detected}\pysigline{\sphinxbfcode{detected}}
\emph{bool} -- True if motion is detected

\end{fulllineitems}

\index{working (rpicameramon.motion.CaptureHandler attribute)}

\begin{fulllineitems}
\phantomsection\label{rpicameramon:rpicameramon.motion.CaptureHandler.working}\pysigline{\sphinxbfcode{working}}
\emph{bool} -- True if the picture is saving

\end{fulllineitems}

\index{i (rpicameramon.motion.CaptureHandler attribute)}

\begin{fulllineitems}
\phantomsection\label{rpicameramon:rpicameramon.motion.CaptureHandler.i}\pysigline{\sphinxbfcode{i}}
\emph{int} -- counter of captured photos

\end{fulllineitems}

\index{echoCounter (rpicameramon.motion.CaptureHandler attribute)}

\begin{fulllineitems}
\phantomsection\label{rpicameramon:rpicameramon.motion.CaptureHandler.echoCounter}\pysigline{\sphinxbfcode{echoCounter}}
\emph{int} -- counter of taken pictures with echo mode

\end{fulllineitems}

\index{motion\_detected() (rpicameramon.motion.CaptureHandler method)}

\begin{fulllineitems}
\phantomsection\label{rpicameramon:rpicameramon.motion.CaptureHandler.motion_detected}\pysiglinewithargsret{\sphinxbfcode{motion\_detected}}{}{}
\end{fulllineitems}

\index{tick() (rpicameramon.motion.CaptureHandler method)}

\begin{fulllineitems}
\phantomsection\label{rpicameramon:rpicameramon.motion.CaptureHandler.tick}\pysiglinewithargsret{\sphinxbfcode{tick}}{}{}
This tick method provides a handler for capturing the pictures.
It ticks every second after PiMotion.start() was called.
If detected is True and method is not processing any capture
(That is indicated by variable `working'), it begins with processing.
First, datetime method is called to obtain the actual datetime, then
the scene is analyzed with scan\_day method which returns true if
light conditions appear to be daylight or false if the light level
is too low.

If the echo mode is activated
\begin{description}
\item[{For daylight:}] \leavevmode
exposure compensation is set to 0 \textless{}-25;25\textgreater{}
exposure mode is set to auto
camera shutter speed is set to 0 (auto mode)

\item[{For bad light conditions:}] \leavevmode
exposure compensation is set to 25 for maximum brightness
exposure mode is set to night preview
shutter speed is set to 200000 microseconds which is equal
to 0,2s
Turn on the LED

\end{description}

\end{fulllineitems}

\index{scan\_day() (rpicameramon.motion.CaptureHandler method)}

\begin{fulllineitems}
\phantomsection\label{rpicameramon:rpicameramon.motion.CaptureHandler.scan_day}\pysiglinewithargsret{\sphinxbfcode{scan\_day}}{}{}
This method captures picture as an RGB array and calculates
average value of the pixels in the matrix.
If the value pixAverage is more than 50 (scene is gray to black).
It indicates that the light condition is poor and we can set up night
params.

\end{fulllineitems}


\end{fulllineitems}

\index{SMSHandler (class in rpicameramon.motion)}

\begin{fulllineitems}
\phantomsection\label{rpicameramon:rpicameramon.motion.SMSHandler}\pysiglinewithargsret{\sphinxstrong{class }\sphinxcode{rpicameramon.motion.}\sphinxbfcode{SMSHandler}}{\emph{handler}}{}
Bases: \sphinxcode{object}
\index{handle\_sms() (rpicameramon.motion.SMSHandler method)}

\begin{fulllineitems}
\phantomsection\label{rpicameramon:rpicameramon.motion.SMSHandler.handle_sms}\pysiglinewithargsret{\sphinxbfcode{handle\_sms}}{\emph{sms}}{}
\end{fulllineitems}


\end{fulllineitems}

\index{PiMotion (class in rpicameramon.motion)}

\begin{fulllineitems}
\phantomsection\label{rpicameramon:rpicameramon.motion.PiMotion}\pysiglinewithargsret{\sphinxstrong{class }\sphinxcode{rpicameramon.motion.}\sphinxbfcode{PiMotion}}{\emph{post\_capture\_callback=None}, \emph{q=None}}{}
Bases: \sphinxcode{object}

This class handles the camera and PIR sensor setup.
It sets up resolution and framerate and starts
capturing the video outputting it to the MotionAnalysis object.
\begin{quote}\begin{description}
\item[{Parameters}] \leavevmode\begin{itemize}
\item {} 
\textbf{\texttt{post\_capture\_callback}} ({\hyperref[rpicameramon:rpicameramon.motion.CaptureHandler.callback]{\sphinxcrossref{\emph{\texttt{callback}}}}}) -- not in use

\item {} 
\textbf{\texttt{q}} (\emph{\texttt{Queue obj}}) -- queue for captured photos

\end{itemize}

\item[{Raises}] \leavevmode
\sphinxcode{KeyboardInterrupt} -- Interrupt the program

\end{description}\end{quote}
\index{start() (rpicameramon.motion.PiMotion method)}

\begin{fulllineitems}
\phantomsection\label{rpicameramon:rpicameramon.motion.PiMotion.start}\pysiglinewithargsret{\sphinxbfcode{start}}{}{}
\end{fulllineitems}


\end{fulllineitems}



\chapter{telemetry module}
\label{rpicameramon:telemetry-module}
This module provides functionality for sending telemetry data to the Google Sheets.
\phantomsection\label{rpicameramon:module-rpicameramon.telemetry}\index{rpicameramon.telemetry (module)}\index{GoogleHandler (class in rpicameramon.telemetry)}

\begin{fulllineitems}
\phantomsection\label{rpicameramon:rpicameramon.telemetry.GoogleHandler}\pysigline{\sphinxstrong{class }\sphinxcode{rpicameramon.telemetry.}\sphinxbfcode{GoogleHandler}}
Bases: \sphinxcode{object}
\index{get\_credentials() (rpicameramon.telemetry.GoogleHandler method)}

\begin{fulllineitems}
\phantomsection\label{rpicameramon:rpicameramon.telemetry.GoogleHandler.get_credentials}\pysiglinewithargsret{\sphinxbfcode{get\_credentials}}{}{}
Gets valid user credentials from storage.

If nothing has been stored, or if the stored credentials are invalid,
the OAuth2 flow is completed to obtain the new credentials.
\begin{quote}\begin{description}
\item[{Returns}] \leavevmode
Credentials, the obtained credential.

\end{description}\end{quote}

\end{fulllineitems}

\index{get\_sheets\_service() (rpicameramon.telemetry.GoogleHandler method)}

\begin{fulllineitems}
\phantomsection\label{rpicameramon:rpicameramon.telemetry.GoogleHandler.get_sheets_service}\pysiglinewithargsret{\sphinxbfcode{get\_sheets\_service}}{\emph{credentials}}{}
Gets Google Sheets Service v4
\begin{quote}\begin{description}
\item[{Returns}] \leavevmode
Google Sheets service object

\end{description}\end{quote}

\end{fulllineitems}

\index{get\_file\_service() (rpicameramon.telemetry.GoogleHandler method)}

\begin{fulllineitems}
\phantomsection\label{rpicameramon:rpicameramon.telemetry.GoogleHandler.get_file_service}\pysiglinewithargsret{\sphinxbfcode{get\_file\_service}}{\emph{credentials}}{}
Gets Google Drive Service v3
\begin{description}
\item[{Retruns:}] \leavevmode
Google Drive service object

\end{description}

\end{fulllineitems}

\index{add\_sheet\_line() (rpicameramon.telemetry.GoogleHandler method)}

\begin{fulllineitems}
\phantomsection\label{rpicameramon:rpicameramon.telemetry.GoogleHandler.add_sheet_line}\pysiglinewithargsret{\sphinxbfcode{add\_sheet\_line}}{\emph{service=None}, \emph{line=None}, \emph{spreadsheetId=None}, \emph{rangeName=None}}{}
Append a line/lines to Google Sheet.
\begin{quote}\begin{description}
\item[{Parameters}] \leavevmode\begin{itemize}
\item {} 
\textbf{\texttt{service}} (\emph{\texttt{obj}}) -- Google Sheets Service object

\item {} 
\textbf{\texttt{line}} (\emph{\texttt{nested list}}) -- matrix of data to be put into Google Sheet

\item {} 
\textbf{\texttt{spreadsheetId}} (\emph{\texttt{str}}) -- Id of spreadsheet

\item {} 
\textbf{\texttt{rangeName}} (\emph{\texttt{str}}) -- Range in a spreadsheet

\end{itemize}

\end{description}\end{quote}

\end{fulllineitems}

\index{upload\_file() (rpicameramon.telemetry.GoogleHandler method)}

\begin{fulllineitems}
\phantomsection\label{rpicameramon:rpicameramon.telemetry.GoogleHandler.upload_file}\pysiglinewithargsret{\sphinxbfcode{upload\_file}}{\emph{service}, \emph{filename}}{}
Uploads a image/jpeg file to the Google Drive.
\begin{quote}\begin{description}
\item[{Parameters}] \leavevmode\begin{itemize}
\item {} 
\textbf{\texttt{service}} (\emph{\texttt{obj}}) -- Google Drive Service object

\item {} 
\textbf{\texttt{filename}} (\emph{\texttt{str}}) -- full path to the file.

\end{itemize}

\end{description}\end{quote}

\end{fulllineitems}


\end{fulllineitems}

\index{LogSender (class in rpicameramon.telemetry)}

\begin{fulllineitems}
\phantomsection\label{rpicameramon:rpicameramon.telemetry.LogSender}\pysiglinewithargsret{\sphinxstrong{class }\sphinxcode{rpicameramon.telemetry.}\sphinxbfcode{LogSender}}{\emph{logQueue}, \emph{googleHandler}}{}
Bases: \sphinxcode{threading.Thread}

LogSender class is a subclass of a Thread class.
It provides a functionality to send a captured log to
Google Sheets. It should be used together with the
logging.handlers.QueueHandler class as a handler for logging.
LogSender works only in specified time intervals,
dequeues everything in logQueue, sends it
to the Google Sheets and goes sleep for the amount of time
specified by the variable fileUploadSleep
\index{logQueue (rpicameramon.telemetry.LogSender attribute)}

\begin{fulllineitems}
\phantomsection\label{rpicameramon:rpicameramon.telemetry.LogSender.logQueue}\pysigline{\sphinxbfcode{logQueue}}
\emph{obj} -- queue.Queue object with logRecord (obj)

\end{fulllineitems}

\index{enqueued (rpicameramon.telemetry.LogSender attribute)}

\begin{fulllineitems}
\phantomsection\label{rpicameramon:rpicameramon.telemetry.LogSender.enqueued}\pysigline{\sphinxbfcode{enqueued}}
\end{fulllineitems}

\index{googleHandler (rpicameramon.telemetry.LogSender attribute)}

\begin{fulllineitems}
\phantomsection\label{rpicameramon:rpicameramon.telemetry.LogSender.googleHandler}\pysigline{\sphinxbfcode{googleHandler}}
\emph{obj} -- GoogleHandler instance object

\end{fulllineitems}

\index{run() (rpicameramon.telemetry.LogSender method)}

\begin{fulllineitems}
\phantomsection\label{rpicameramon:rpicameramon.telemetry.LogSender.run}\pysiglinewithargsret{\sphinxbfcode{run}}{}{}
\end{fulllineitems}


\end{fulllineitems}

\index{TelemetrySender (class in rpicameramon.telemetry)}

\begin{fulllineitems}
\phantomsection\label{rpicameramon:rpicameramon.telemetry.TelemetrySender}\pysiglinewithargsret{\sphinxstrong{class }\sphinxcode{rpicameramon.telemetry.}\sphinxbfcode{TelemetrySender}}{\emph{googleHandler}}{}
Bases: \sphinxcode{threading.Thread}

Telemetry class is a subclass of a Thread class.
It provides a functionality for sending a telemetric data to
Google Sheets.

Attributes:
\index{run() (rpicameramon.telemetry.TelemetrySender method)}

\begin{fulllineitems}
\phantomsection\label{rpicameramon:rpicameramon.telemetry.TelemetrySender.run}\pysiglinewithargsret{\sphinxbfcode{run}}{}{}
\end{fulllineitems}

\index{getTelemetry() (rpicameramon.telemetry.TelemetrySender method)}

\begin{fulllineitems}
\phantomsection\label{rpicameramon:rpicameramon.telemetry.TelemetrySender.getTelemetry}\pysiglinewithargsret{\sphinxbfcode{getTelemetry}}{}{}
\end{fulllineitems}


\end{fulllineitems}




\renewcommand{\indexname}{Index}
\printindex
\end{document}
