%% Název práce:
%  První parametr je název v originálním jazyce,
%  druhý je překlad v angličtině nebo češtině (pokud je originální jazyk angličtina)
\nazev{Kamerový zapezpečovací systém s minipočítačem Raspberry Pi}{Camera Security System with Raspberry Pi Minicomputer}

%% Jméno a příjmení autora ve tvaru
%  [tituly před jménem]{Křestní}{Příjmení}[tituly za jménem]
\autor{Marek}{Vitula}

%% Jméno a příjmení vedoucího/školitele včetně titulů
%  [tituly před jménem]{Křestní}{Příjmení}[tituly za jménem]
% Pokud osoba nemá titul za jménem, smažte celý řetězec '[...]'
\vedouci[doc.\ Ing.]{Jaromír}{Kolouch}[CSc.]

%% Jméno a příjmení oponenta včetně titulů
%  [tituly před jménem]{Křestní}{Příjmení}[tituly za jménem]
% Pokud nemá titul za jménem, smažte celý řetězec '[...]'
% Uplatní se pouze v prezentaci k obhajobě;
% v případě, že nechcete, aby se na titulním snímku prezentace zobrazoval oponent, pouze jej zakomentujte;
% u obhajoby semestrální práce se oponent nezobrazuje
\oponent[doc.\ Mgr.]{Křestní}{Příjmení}[Ph.D.]

%% Označení oboru studia
% První parametr je obor v originálním jazyce,
% druhý parametr je překlad v angličtině nebo češtině
\oborstudia{Elektronika a sdělovací technika}{Electronics and Communications}

%% Označení fakulty
% První parametr je název fakulty v originálním jazyce,
% druhý parametr je překlad v angličtině nebo v češtině
%\fakulta{Fakulta architektury}{Faculty of Architecture}
\fakulta{Fakulta elektrotechniky a komunikačních technologií}{Faculty of Electrical Engineering and Communication}
%\fakulta{Fakulta chemická}{Faculty of Chemistry}
%\fakulta{Fakulta informačních technologií}{Faculty of Information Technology}
%\fakulta{Fakulta podnikatelská}{Faculty of Business and Management}
%\fakulta{Fakulta stavební}{Faculty of Civil Engineering}
%\fakulta{Fakulta strojního inženýrství}{Faculty of Mechanical Engineering}
%\fakulta{Fakulta výtvarných umění}{Faculty of Fine Arts}

%% Označení ústavu
% První parametr je název ústavu v originálním jazyce,
% druhý parametr je překlad v angličtině nebo češtině
%\ustav{Ústav automatizace a měřicí techniky}{Department of Control and Instrumentation}
%\ustav{Ústav biomedicínského inženýrství}{Department of Biomedical Engineering}
%\ustav{Ústav elektroenergetiky}{Department of Electrical Power Engineering}
%\ustav{Ústav elektrotechnologie}{Department of Electrical and Electronic Technology}
%\ustav{Ústav fyziky}{Department of Physics}
%\ustav{Ústav jazyků}{Department of Foreign Languages}
%\ustav{Ústav matematiky}{Department of Mathematics}
%\ustav{Ústav mikroelektroniky}{Department of Microelectronics}
\ustav{Ústav radioelektroniky}{Department of Radio Electronics}
%\ustav{Ústav teoretické a experimentální elektrotechniky}{Department of Theoretical and Experimental Electrical Engineering}
%\ustav{Ústav telekomunikací}{Department of Telecommunications}
%\ustav{Ústav výkonové elektrotechniky a elektroniky}{Department of Power Electrical and Electronic Engineering}

\logofakulta[loga/FEKT_zkratka_barevne_PANTONE_CZ]{loga/UTKO_color_PANTONE_CZ}


%% Rok obhajoby
\rok{2018}
\datum{13.\,12.\,2017} % Datum se uplatní pouze v prezentaci k obhajobě

%% Místo obhajoby
% Na titulních stránkách bude automaticky vysázeno VELKÝMI písmeny
\misto{Brno}

%% Abstrakt
\abstrakt{Tato bakalářská práce se zabývá návrhem kamerového zabezpečovacího systému na platformě Raspberry Pi. V první části popisuje jednotlivé hardwarové a softwarové komponenty, které budou použity pro zařízení. V druhé části se práce zabývá návrhem aplikace pro detekci pohybu z obrazu a též pomocí pohybového čidla PIR a následného zpracování dat a odeslání na cloudové úložiště. Závěrečná část práce se věnuje hardwarové části, především návrhu napájení zařízení akumulátorem, který je dobíjen ze solárního panelu a také návrhu tranzistorového spínače pro výkonovou LED.}
{This bachelor thesis describes a design of a camera surveillance system on the Raspbery Pi platform. First part of the thesis describes the individual hardware and software components to be used for the device. Second part deals with the design of the actual application for motion detection using the output from the camera and PIR motion sensor and the following processing and sending data to a cloud storage. The final part of the thesis describes the design of the hardware, particularly solar panel and battery and also the design of the power LED switch.}

%% Klíčová slova
\klicovaslova{Raspberry Pi, kamera, počítačové vidění, solární napájení, Python, Linux }%
	{Raspberry Pi, camera, computer vision, solar powered, Python, Linux}

%% Poděkování
\podekovanitext{Rád bych poděkoval vedoucímu bakalářské práce panu doc. Ing. Jaromírovi Kolouchovi, CSc.\ a odbornému konzultantovi panu Ing. Ondřejovi Pavelkovi\ za odborné vedení, trpělivost při konzultacích a jejich podnětné návrhy k~mé práci.}