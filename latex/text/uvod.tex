\chapter*{Úvod}
\phantomsection
\addcontentsline{toc}{chapter}{Úvod}


Tento dokument popisuje semestrální práci vytvořenou na VUT, fakultě elektrotechniky, v Brně.

Po krátkém úvodu a seznámení se zadáním práce následuje kapitola číslo 1, ve které je popsán minipočítač Raspberry Pi, na kterém bude práce dále stavět. Dále je popsán operační systém Raspbian a další 
programové vybavení, které bude použito. Je zde teoreticky popsán způsob detekce pohybu a také je teoreticky objasněn problém solárního napájení.

Kapitola č. 2 popisuje návrh aplikace pro detekci pohybu.

Návrh řešení se nachází v kapitole č. 3 a popis samotné realizace projektu je v kapitole č. 4.  

Poslední kapitola krátce hodnotí celou práci a navrhuje případná zlepšení.

\section*{Úvod do problematiky}
Práce si klade za cíl vytvořit zařízení, které bude schopné sloužit jako monitorovací systém, například pro ostrahu objektu, nebo pro monitorování zvěře.

Ve srovnání s komerčně dostupnými, kamerovými zabezpečovacími systémy, bude navrhovaný systém, vytvořený na platformě Raspberry Pi a Linux, disponovat větší variabilitou.

Monitorovací systém je navržen tak, aby mohl být po nějakou dobu energeticky soběstačný. Problém lze řešit napájením pomocí solárního panelu, kterým bude dobíjen akumulátor.

Systém by tedy měl být energeticky úsporný. Bude nutné vybrat vhodnou, spolehlivou metodu detekce obrazu, která by nevyžadovala pro svoji činnost příliš mnoho elektrické energie.

Konektivita monitorovacího systému do sítě Internet bude zajištěna pomocí přídavného modulu. 